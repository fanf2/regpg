% This file is included by notes.tex and slides.tex

\usepackage[german,greek,english,british]{babel}
\usepackage{pgf}
\usepackage{color}
\usepackage{eurosym}
\usepackage{hyperref}
\usepackage{pifont} % dingbats
\usepackage{calc} % for frame column width calculations
\usepackage{tikz}
\usepackage[normalem]{ulem} % for strikeout
\usepackage{url}
\usepackage[utf8x]{inputenc}

\frenchspacing
\setlength{\parindent}{0pt}

\DeclareUrlCommand\email{%
  \def\UrlLeft{<}%
  \def\UrlRight{>}%
  \urlstyle{tt}%
}
\newcommand\code[1]{\texttt{#1}}

\newcommand\notes[1]{\mode<article>{#1}}

\newcommand\hardbreak{\notes{\pagebreak}}

\newcommand\gap[1]{\mode<beamer>{\vspace{#1}}}

\newcommand\git{\code{git}}
\newcommand\gpg{\code{gpg}}
\newcommand\regpg{\code{regpg}}
\newcommand\ssh{\code{ssh}}

% args: size, file, URL
\newcommand\image[3][]{
  \mode<beamer>{
    \includegraphics[#1]{#2}\\
    \tiny\textcolor{gray}{\url{#3}}
  }
}

% args: title, body
\newcommand\slide[2]{
  \frame{\frametitle{#1}#2}
}

% args: (optional) image width, image file, image URL, slide title
\newcommand\slideI[4][10cm]{
  \slide{#4}{
    \centering
    \image[width=#1]{#2}{#3}
  }
}

%%%%%%%%%%%%%%%%%%%%%%%%%%%%%%%%%%%%%%%%%%%%%%%%%%%%%%%%%%%%%%%%%%%%%%%%

% common metadata

\author[Tony~Finch]{
  Tony~Finch \\
  \email{fanf2@cam.ac.uk} \\
  \email{gitmaster@uis.cam.ac.uk}
}

\institute{%
  Network Systems (RNB 1N52)%
}

\titlegraphic{%
  \raisebox{1pt}{%
    \includegraphics[height=20pt]{univ.jpg}%
  }\\
  University Information Services%
}

%%%%%%%%%%%%%%%%%%%%%%%%%%%%%%%%%%%%%%%%%%%%%%%%%%%%%%%%%%%%%%%%%%%%%%%%

% title page

\title{
  \regpg
}
\subtitle{
  safely store server secrets
}
\date{Tuesday 21st November 2017}

\begin{document}

\frame{\maketitle}

\notes{
  \begin{abstract}
    The \regpg\ program is a thin wrapper around \gpg\ for encrypting
    secrets so they can be stored and shared using \git\ and decrypted
    when Ansible deploys them to servers.
  \end{abstract}
}

%%%%%%%%%%%%%%%%%%%%%%%%%%%%%%%%%%%%%%%%%%%%%%%%%%%%%%%%%%%%%%%%%%%%%%%%

\section{Introduction}

\notes {
  This talk is in two main sections.
}

\slide{agenda}{
  \begin{columns}[T]

    \notes{
      I will start off by explaining some of the context and thinking
      behind \regpg\ by unpacking its slogan backwards.
    }

    \column{5cm}
    \begin{itemize}
    \Large
    \item Context
      \begin{itemize}
      \Large
      \item secrets?
      \item server?
      \item store?
      \item safely?
      \item gpg?
      \item re?
      \end{itemize}
    \end{itemize}

    \notes{
      Then I'll give a demo of \regpg's main features, in roughly the same
      order as its reference manual.
    }

    \column{5cm}
    \begin{itemize}
    \Large
    \item Demo
      \begin{itemize}
      \Large
      \item keys
      \item secrets
      \item recrypt
      \item X.509 / TLS
      \item Ansible
      \item conversion
      \end{itemize}
    \end{itemize}

  \end{columns}
}

%%%%%%%%%%%%%%%%%%%%%%%%%%%%%%%%%%%%%%%%%%%%%%%%%%%%%%%%%%%%%%%%%%%%%%%%

\section{Context}

\notes{

  In the first part, we'll discuss what \regpg\ is and what it is not.
}

\slideI[6cm]{keys.jpg}{
  https://www.flickr.com/photos/fuzzy/3196534149
}{secrets}

\notes{

  The secrets we are working with are cryptographic keys
  \begin{itemize}
  \item private keys
  \item bearer tokens
  \item shared secrets
  \end{itemize}

  We have hundreds of them. They need to be shared with the right
  people and kept secret from the wrong people.

  It's a key distribution problem.
}

\slideI[8cm]{encrypt-all-the-things.png}{
  https://richardskingdom.net/wp-content/uploads/2014/04/encrypt-all-the-things.png
}{secrets -- encryption}

\notes{

  We can massively reduce the size of the problem by encrypting
  the secrets with a small number of master secrets.

  For example, before \regpg\ I used to encrypt secrets using the root
  password.

  This reduces the key distribution to previously solved problems:

  \begin{itemize}
  \item password distribution
  \item non-secret file distribution (i.e. git)
  \end{itemize}

}

\slideI[10cm]{shamir-rivest-adleman.jpg}{
  https://claudiodinardo.com/content/images/2017/08/shamir-rivest-adleman.jpg
}{secrets -- Shamir / Rivest / Adleman}

\notes{

  But we can do better with public key cryptography.

  Each person keeps their own private key -- there's no need to
  distribute any master secrets. We know how to do this because we
  already do it for \ssh keys.

  We distribute the public keys of each person who can decrypt, which
  gives us a kind of auditable record of who has access to secrets.

  We can revoke a person's access if we can destroy all the copies of
  their private key, without having to replace all the secrets.

  You only need the public keys to encrypt a secret, which means an
  automated system can manage its own keys without having access to
  all the other secrets in a repository.

}

\slideI[8cm]{server.jpg}{
  https://www.flickr.com/photos/evilnick/183967344
}{server}

\notes{

  The specific kinds of secrets we are dealing with are used by
  servers to authenticate themselves --

  \begin{itemize}
  \item \ssh host private keys
  \item TLS private keys
  \item API keys
  \item DNS TSIG shared secrets
  \item etc.
  \end{itemize}

  These secrets have to be available unencrypted on the server, so we
  want it to be convenient to decrypt and install them.

  We're not dealing with user passwords.

  We're not trying to be a password manager.

}

\slideI[9cm]{files.jpg}{
  https://www.flickr.com/photos/lnx/7297731
}{server -- files}

\notes{

  It's often the case that each server secret is in a file by itself --
  that's true for \ssh and TLS and DNS keys.

  \regpg\ works best when each secret is in a file by itself. You can
  use filenames to identify secrets without having to decrypt them.

  Keeping secrets strictly separate from non-secret code and
  configuration helps \git\ \code{diff} to works better.

  \regpg\ does not have any hooks into \git\ for automatically
  decrypting and \code{diff}ing secrets because secrets are blobs of
  random data for which \code{diff} is useless.

}

\slideI[11cm]{library.jpg}{
  https://www.flickr.com/photos/23605686@N05/6921691127
}{store -- not share}

\notes{

  \regpg\ is for encrypting files for storage in version control when
  they are not in use, and decrypting them for deployment to
  production.

  The other verb that might have fitted in this place is ``share'',
  but \regpg\ is not directly about sharing.

  \regpg\ stores secrets in a way that works with \git\ or other version
  control systems, but \regpg\ does not get involved with \git. You use
  \git\ for sharing secrets in the same way you us it for sharing code
  or configuration.

  I have tried a few times to write wrappers that get clever with
  \git\ and they have usually been dismal failures. \regpg\ does
  not get clever with \git.

}

\slideI[5cm]{hazmat.jpg}{
  https://www.flickr.com/photos/mamboman/3698344360
}{safely -- hazmat containment}

\notes{

  There are a couple of aspects to being safe with \regpg, and both of
  them relate to dissatisfaction with \code{ansible-vault}.

  The first is safe cryptography.

  \regpg\ keeps well away from any low-level primitives. I did a code
  review of \code{ansible-vault} and it uses a cryptographic library
  that literally has ``HAZMAT'' in its name. And, totally predictably,
  \code{ansible-vault} has really bad crypto.

  Instead, \regpg\ relies on \gpg\ for cryptography. \gpg\ is terrible
  software in many ways, but it is widely available, it has reasonably
  competent crypto, and it is also used by \git\ and Debian.

}

\slideI[10cm]{checklist.jpg}{
  https://www.flickr.com/photos/109570752@N02/15118828431
}{safely -- situational awareness}

\notes{

  The other aspect to being safe is psychological safety.

  \regpg\ allows you to make it clear in your Ansible playbook which
  files should be encrypted, helps you to find out which files
  actually are or are not encrypted, and tells you when things are
  inconsistent.

  This is unlike \code{ansible-vault} which does not let you say
  whether something should be encrypted, and encourages you to encrypt
  and decrypt in place, and doesn't complain either way, so you can
  easily expose secrets by mistake.

  \regpg\ tries to be really easy to understand. It isn't very chatty,
  but it also does not hide things from you. I want you to feel
  confident that you know how it works and what it is doing.

}

\slideI[10cm]{gnupg.png}{
  https://commons.wikimedia.org/wiki/File:Gnupg_logo.svg
}{gpg}

\notes{

  \regpg\ is a thin wrapper around \gpg\ to adapt it for our purposes.

  It's a very thin wrapper. You don't need \regpg\ to decrypt secrets
  -- you can still use normal \code{gpg -d} to decrypt them.

  \regpg\ simplifies \gpg\ in two ways.

  Firstly, \regpg\ gets rid of \gpg's key management and replaces it
  with Jon Warbrick's scheme.

  There are no key servers, no web of trust, no key signing parties.
  Instead we just use \git\ to exchange public keys.

  Secondly, \regpg\ provides several little helpers to make it easier
  to use \gpg-encrypted secrets with other tools such as OpenSSL,
  OpenSSH, and Ansible.

}

\slideI[8cm]{re3d.png}{
  https://dotat.at/prog/regpg/
}{regpg}

\notes{

  Why is it called ``\regpg''? Where does the ``re'' come from?

  Partly named after its \code{recrypt} subcommand which we will see
  shortly.

  In \gpg, the term ``recipients'' means those who can decrypt a
  message. \regpg\ is all about managing a list of recipients and
  repeatably and reliably encrypting files to those recipients.

}

%%%%%%%%%%%%%%%%%%%%%%%%%%%%%%%%%%%%%%%%%%%%%%%%%%%%%%%%%%%%%%%%%%%%%%%%

\section{Demo}

\slide{dependencies}{
  \begin{columns}[T]

    \column{5cm}
    \begin{itemize}
    \Large
    \item prerequisites
      \begin{itemize}
      \Large
      \item perl
      \item gnupg
      \item gnupg-agent
      \item pinentry-*
      \end{itemize}
    \end{itemize}

    \column{5cm}
    \begin{itemize}
    \Large
    \item helpers
      \begin{itemize}
      \Large
      \item ansible
      \item git
      \item openssl
      \item openssh-client
      \item xclip
      \end{itemize}
    \end{itemize}

  \end{columns}
}

\notes{

  The prerequisites are required for \regpg's core functionality; the
  helpers are optional but some \regpg\ features won't work without
  them.

  These are Debian package names. If anyone is able to help with
  installation instructions on other systems, please let me know!

  The \code{pinentry} program is used by \code{gpg-agent} to prompt
  you for your passphrase. There are multiple versions -- I use
  \code{pinentry-gtk2} but there are also \code{-gnome3} and
  \code{-qt} and \code{-curses} versions.
}

\slide{check \code{gpg-agent}}{

  \Large

  \begin{itemize}
  \item \code{echo \$GPG\char`_AGENT\char`_INFO }
  \item \code{eval \$(gpg-agent --daemon) }
  \end{itemize}

}

\notes{

  You should find that \code{gnupg-agent} is started automatically
  when you log in - use the first command to check this.

  You can start it manually using the second command.

}

\slide{install}{


  \begin{itemize}
    \large
  \item quick
    \begin{itemize}
      \large
    \item \code{cd \char`~/bin}
    \item \code{curl -O https://dotat.at/prog/regpg/regpg}
    \end{itemize}
  \end{itemize}

  \begin{itemize}
    \large
  \item home page \url{https://dotat.at/prog/regpg/}
    \begin{itemize}
      \large
    \item supporting documentation
    \item distribution tar balls
    \item test suite
    \end{itemize}
  \end{itemize}

}

\slide{generate key}{

  \Large

  \begin{itemize}
  \item Generate a key just for \regpg
  \item Separate from your other \gpg\ keys (if any)
  \end{itemize}

  \begin{itemize}
  \item \code{gpg --gen-key}
  \end{itemize}

  \begin{itemize}
  \item Answer the quiz
  \end{itemize}
}

\slideI[10cm]{crash1.jpg}{
  https://www.flickr.com/photos/eugenuity/34113551603
}{generate key -- demo}

\slide{manage keys}{

  \Large

  \begin{itemize}
  \item addkey
  \item addself \textcolor{red}{$ \Longleftarrow $}
  \item delkey \textcolor{red}{$ \Longleftarrow $}
  \item exportkey
  \item importkey
  \item lskeys \textcolor{red}{$ \Longleftarrow $}
  \end{itemize}
}

\slideI[10cm]{crash2.jpg}{
  https://www.flickr.com/photos/bantam10/3068761016
}{manage keys -- demo}

\notes{

  \begin{itemize}
  \item mkdir demo
  \item cd demo
  \item regpg addself
  \item ls
  \end{itemize}

  \regpg\ has made a public key ring (and a backup file,
  because \gpg\ loves backup files)

  The ``addself'' subcommand adds keys which match your login name and
  for which you have the private key.

  This is the only configuration file for \regpg

  Normally you would put this at the top of your Ansible setup next to
  your \code{ansible.cfg} and inventory etc.

  \begin{itemize}
  \item regpg lskeys
  \item regpg ls
  \item regpg del fanf9
  \item regpg ls
  \item regpg add fanf9
  \item regpg ls
  \end{itemize}

  One bit of magic going on here is that \regpg\ ensures that
  \gpg\ uses the backwards compatible keyring format, even if
  you are using \gpg\ 2.1.

}

\slide{secrets}{

  \Large

  \begin{itemize}
  \item encrypt \textcolor{red}{$ \Longleftarrow $}
  \item decrypt \textcolor{red}{$ \Longleftarrow $}
  \item recrypt
  \item edit \textcolor{red}{$ \Longleftarrow $}
  \item pbcopy
  \item pbpaste
  \item shred \textcolor{red}{$ \Longleftarrow $}
  \end{itemize}

  \begin{itemize}
  \item check \textcolor{red}{$ \Longleftarrow $}
  \end{itemize}

}

\notes{

  The pasteboard commands use the names from Mac OS X, but if you use
  them on Linux it will use \code{xclip} instead.

}

\slideI[10cm]{crash3.jpg}{
  https://www.flickr.com/photos/zapthedingbat/516726771
}{secrets -- demo}

\notes{

  \begin{itemize}
  \item echo secret one >foo
  \item regpg encrypt foo foo.asc
  \end{itemize}

  Note \regpg\ does not need a passphrase to encrypt, just the public keys.

  The \code{.asc} extension is the ugly but conventional name for a
  PGP-encrypted ASCII-armored file. (ASCII armoring is like Base64.)

  \begin{itemize}
  \item regpg check
  \end{itemize}

  The ``check'' subcommand looks for encrypted files by recursively
  grepping for the ``BEGIN PGP MESSAGE'' ASCII-armoring.

  \regpg\ warns us that we have left behind an unencrypted file.
  (It uses a simple heuristic based on filenames.)

  \begin{itemize}
  \item regpg shred foo
  \item regpg ck
  \item regpg decrypt foo.asc
  \end{itemize}

  \regpg\ asks for your passphrase to decrypt the first time

  \begin{itemize}
  \item regpg decrypt foo.asc
  \end{itemize}

  The \code{gpg-agent} has stashed the passphrase so we don't need to
  keep typing it.

  \begin{itemize}
  \item regpg edit foo.asc
  \end{itemize}

  You should not normally need to edit an encrypted file by hand, but
  if you do, \regpg\ tries to make it safer by keeping temporary files
  in a RAM disk (at least on Linux -- there's no ramfs on Mac OS) and
  shredding them afterwards.

  \begin{itemize}
  \item echo secret three | regpg en bar.asc
  \end{itemize}

  \regpg\ is friendly to pipelines.

  \begin{itemize}
  \item regpg ck
  \item touch foo bar
  \item regpg ck
  \item regpg shred -r
  \item regpg ck
  \end{itemize}

  Several \regpg\ subcommands take a \code{-r} option which recurses
  over all the files found by \code{regpg check}

}

\slide{recrypt}{

  \Large

  \begin{itemize}
  \item delkey \textcolor{red}{$ \Longleftarrow $}
  \item importkey \textcolor{red}{$ \Longleftarrow $}
  \item lskeys \textcolor{red}{$ \Longleftarrow $}
  \end{itemize}

  \begin{itemize}
  \item recrypt \textcolor{red}{$ \Longleftarrow $}
  \end{itemize}

  \begin{itemize}
  \item check \textcolor{red}{$ \Longleftarrow $}
  \end{itemize}

}

\slideI[6cm]{crash4.jpg}{
  https://www.flickr.com/photos/parkstreetparrot/6531496943
}{recrypt -- demo}

\notes{

  \begin{itemize}
  \item curl https://dotat.at/fanf.gpg | regpg importkey
  \item regpg ls
  \item regpg ck
  \end{itemize}

  \regpg\ says a key has been added to the keyring and lists which
  files need to be decrypted and re-encrypted so that every recipient
  listed in the keyring can decrypt them

  \begin{itemize}
  \item regpg recrypt -r
  \item regpg ck
  \end{itemize}

  Again the \code{-r} option means recursively apply the command to
  all the files listed by \code{regpg check}

  \begin{itemize}
  \item regpg del fanf2
  \item regpg ls
  \item regpg ck
  \item regpg re -r
  \end{itemize}

  The \code{-r} option can be applied to the key management subcommands
  to combine them with the ``recrypt'' subcommand

  \begin{itemize}
  \item curl https://dotat.at/fanf.gpg | regpg importkey -r
  \item regpg ls
  \item regpg ck
  \item regpg del -r fanf2
  \item regpg ls
  \item regpg ck
  \end{itemize}

}

\slide{generate TLS / ssh}{

  \Large

  \begin{itemize}
  \item gencsrconf \textcolor{red}{$ \Longleftarrow $}
  \item gencsr \textcolor{red}{$ \Longleftarrow $}
  \item genkey \textcolor{red}{$ \Longleftarrow $}
  \item genpwd
  \end{itemize}

}

\notes{

  It turns out that OpenSSL and OpenSSH have mostly the same key
  format, so \regpg\ uses the same ``genkey'' command for both of them.

}

\slideI[10cm]{crash5.jpg}{
  https://www.flickr.com/photos/zapthedingbat/516726771
}{generate TLS / ssh -- demo}

\notes{

  \begin{itemize}
  \item regpg genkey rsa id\char`_rsa.asc id\char`_rsa.pub
  \end{itemize}

  For ssh keypairs, give it a key algorithm, and private and public
  key files.

  \begin{itemize}
  \item regpg genkey rsa tls.pem.asc
  \end{itemize}

  For TLS give it the algorithm and private key file name.

  I always find it difficult to remember how to make a certificate
  signing request, so \regpg\ provides some help.

  First, get a configuration file from an existing certificate, either
  from a file or a web server:

  \begin{itemize}
  \item regpg gencsrconf cam.ac.uk tls.csr.conf
  \item vi tls.csr.conf
  \item regpg gencsr tls.pem.asc tls.csr.conf tls.csr
  \end{itemize}

  You should commit your CSR so you can re-use it next time if none of
  the details of yur certificate have changed.

  You should commit your CSR configuration file to keep a convenient
  record of changes to your CSR.

}

\slide{set up hooks}{

  \Large

  \begin{itemize}
  \item init \textcolor{red}{$ \Longleftarrow $}
  \item init git \textcolor{red}{$ \Longleftarrow $}
  \item init ansible \textcolor{red}{$ \Longleftarrow $}
  \item init ansible-vault
  \end{itemize}

}

\notes{

  All these commands are itempotent, and unlike other \regpg\ commands,
  they are quite verbose.

}

\slideI[10cm]{crash6.jpg}{
  https://www.flickr.com/photos/walkingsf/8143196966
}{set up hooks -- demo}

\notes{

  \begin{itemize}
  \item regpg init
  \end{itemize}

  Does nothing if there is a \code{pubring.gpg} file

  \begin{itemize}
  \item rm pubring.gpg*
  \item regpg init
  \end{itemize}

  If there is no keyring, it does ``addself'' verbosely.

  There is a tiny hook for git, which lets you see the history of
  \code{pubring.gpg} more easily.

  (It does not do anything for secret files since it isn't useful to
  diff cryptographic secrets.)

  \begin{itemize}
  \item git init
  \item git add .
  \item git commit -m 'initial commit'
  \item git log --patch pubring.gpg
  \item regpg init git
  \item git log --patch pubring.gpg
  \item git status
  \end{itemize}

  There are two parts to this hook, a \code{.gitattributes} file which
  you should commit, and some local repository configuration which
  cannot be committed.

  \begin{itemize}
  \item git add .gitattributes
  \item git commit -m 'regpg init git'
  \end{itemize}

  Whenever you newly clone a repository, you should run
  \code{regpg init git} inside it to set up the local configuration.

  \begin{itemize}
  \item curl https://dotat.at/fanf.gpg | regpg importkey
  \item git diff
  \end{itemize}

  This feature is really important for auditing changes to your
  \code{pubring.gpg} file, because that's your access control list.

  OK, let's try out Ansible

  \begin{itemize}
  \item echo [defaults] $>$ansible.cfg
  \item echo hostfile = inventory $>>$ansible.cfg
  \item echo localhost ansible\texttt{\char`_}connection=local $>$inventory
  \item ansible -m debug -a msg=hi localhost
  \item git add .
  \item git commit -m 'start ansible'
  \end{itemize}

  OK, we have a basic Ansible setup.

  \begin{itemize}
  \item regpg init ansible
  \item cat ansible.cfg
  \end{itemize}

  \regpg\ added a plugin for decrypting secrets. This is 20 lines of
  python that just invokes \code{gpg --decrypt} -- the plugin does not
  use \regpg.

  \begin{itemize}
  \item cat gpg-preload.yml
  \end{itemize}

  \regpg\ added a plugin for preloading \code{gpg-agent}. You can use
  this to make gpg ask you for your passphrase just once -- there is a
  race condition which can make it asking for every host.

  \begin{itemize}
  \item ansible-playbook gpg-preload.yml
  \end{itemize}

  \begin{itemize}
  \item git add .
  \item git commit -m 'regpg init ansible'
  \end{itemize}

  There's an example of how to use this setup at the end of the
  \regpg\ man page.

  \begin{itemize}
  \item regpg help
  \end{itemize}

  The thing to note here is that I am using
  \code{with\char`_fileglob:} to make ansible search for files using
  its usual search path. The \code{gpg\char`_d} plugin does no
  searching.

  I am using this setup at the moment for my systems. It's optimized
  for simplicity of implementation, though it can be a bit annoying.
  I'm interested in feedback if you think something more elaborate
  would be worth the effort.

}

\slide{converters}{

  \Large

  \begin{itemize}
  \item conv ansible-gpg \textcolor{red}{$ \Longleftarrow $}
  \item conv ansible-vault \textcolor{red}{$ \Longleftarrow $}
  \item conv stgza
  \end{itemize}

}

\slideI[6cm]{xform.jpg}{
  https://www.flickr.com/photos/eltpics/15367149536
}{converters -- demo}

\notes{

  I'm going to set up \regpg\ for use with \code{ansible-vault} which
  is only recommended if you want to convert from \code{ansible-vault}
  to \regpg.

  \begin{itemize}
  \item regpg init ansible-vault
  \item git status
  \item cat ansible.cfg
  \end{itemize}

  Now we have a setup similar to Jon Warbrick's \code{ansible-gpg}

  \begin{itemize}
  \item echo hello wombats >wombat
  \item ansible-vault encrypt wombat
  \item cat wombat
  \item vi echidna.yml
  \item ansible-playbook echidna.yml
  \end{itemize}

% ---
% - hosts: all
%   tasks:
%     - debug:
%         msg: "{{ item }}"
%       with\char`_file:
%         - wombat

  Ansible is automatically decrypting things for us.

  Let's convert this file to a normal \regpg\ setup.

  \begin{itemize}
  \item regpg conv ansible-vault
  \item regpg conv ansible-vault wombat wombat.asc
  \item vi echidna.yml
  \end{itemize}

% ---
% - hosts: all
%   tasks:
%     - debug:
%         msg: "{{ item | gpg_d }}"
%       with_fileglob:
%         - wombat.asc

  \begin{itemize}
  \item regpg ck
  \item regpg shred -r
  \end{itemize}

  There's another conversion command which helps with conversion from
  \code{ansible-gpg}

  \begin{itemize}
  \item git clone git://git.uis.cam.ac.uk/uis/u/jw35/ansible-gpg.git
  \item cd ansible-gpg
  \item ls -la
  \item file .ansible-gpg/pubring.gpg
  \end{itemize}

  The \code{ansible-gpg} repo comes with a demo setup. Note that the
  pubring is in incompatible \gpg\ 2.1 keybox format.

  \begin{itemize}
  \item regpg conv ansible-gpg
  \item git status
  \item file pubring.gpg
  \item regpg conv ansible-vault
  \end{itemize}

  This is now the setup that you get from \code{regpg init ansible-vault}
  and it allows you to convert your setup one file at a time.

}


%%%%%%%%%%%%%%%%%%%%%%%%%%%%%%%%%%%%%%%%%%%%%%%%%%%%%%%%%%%%%%%%%%%%%%%%

\section{Done!}

\slideI[9cm]{what.jpg}
       {https://www.flickr.com/photos/debord/4932655275}
       {Questions?}

%

\notes{
  \bibliography{talk}
  \bibliographystyle{plain}
}

\end{document}

% eof
