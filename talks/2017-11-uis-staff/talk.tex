% This file is included by notes.tex and slides.tex

\usepackage[german,greek,english,british]{babel}
\usepackage{pgf}
\usepackage{color}
\usepackage{eurosym}
\usepackage{hyperref}
\usepackage{pifont} % dingbats
\usepackage{calc} % for frame column width calculations
\usepackage{tikz}
\usepackage[normalem]{ulem} % for strikeout
\usepackage{url}
\usepackage[utf8x]{inputenc}

\frenchspacing
\setlength{\parindent}{0pt}

\DeclareUrlCommand\email{%
  \def\UrlLeft{<}%
  \def\UrlRight{>}%
  \urlstyle{tt}%
}
\newcommand\code[1]{\texttt{#1}}

\newcommand\notes[1]{\mode<article>{#1}}

\newcommand\hardbreak{\notes{\pagebreak}}

\newcommand\gap[1]{\mode<beamer>{\vspace{#1}}}

\newcommand\git{\code{git}}
\newcommand\gpg{\code{gpg}}
\newcommand\regpg{\code{regpg}}
\newcommand\ssh{\code{ssh}}

% args: size, file, URL
\newcommand\image[3][]{
  \mode<beamer>{
    \includegraphics[#1]{#2}\\
    \tiny\textcolor{gray}{\url{#3}}
  }
}

% args: title, body
\newcommand\slide[2]{
  \frame{\frametitle{#1}#2}
}

% args: (optional) image width, image file, image URL, slide title
\newcommand\slideI[4][10cm]{
  \slide{#4}{
    \centering
    \image[width=#1]{#2}{#3}
  }
}

%%%%%%%%%%%%%%%%%%%%%%%%%%%%%%%%%%%%%%%%%%%%%%%%%%%%%%%%%%%%%%%%%%%%%%%%

% common metadata

\author[Tony~Finch]{
  Tony~Finch \\
  \email{fanf2@cam.ac.uk} \\
  \email{gitmaster@uis.cam.ac.uk}
}

\institute{%
  Network Systems (RNB 1N52)%
}

\titlegraphic{%
  \raisebox{1pt}{%
    \includegraphics[height=20pt]{univ.jpg}%
  }\\
  University Information Services%
}

%%%%%%%%%%%%%%%%%%%%%%%%%%%%%%%%%%%%%%%%%%%%%%%%%%%%%%%%%%%%%%%%%%%%%%%%

% title page

\title{
  \regpg
}
\subtitle{
  safely store server secrets
}
\date{Tuesday 21st November 2017}

\begin{document}

\frame{\maketitle}

\notes{
  \begin{abstract}
    The \regpg program is a thin wrapper around \gpg for encrypting
    secrets so they can be stored and shared using \git and decrypted
    when Ansible deploys them to servers.
  \end{abstract}
}

%%%%%%%%%%%%%%%%%%%%%%%%%%%%%%%%%%%%%%%%%%%%%%%%%%%%%%%%%%%%%%%%%%%%%%%%

\section{Introduction}

\notes {
  This talk is in two main sections.
}

\slide{agenda}{

  \begin{columns}[T]

    \notes{
      I will start off by explaining some of the context and thinking
      behind \regpg by unpacking its slogan backwards.
    }

    \column{3cm}
    \begin{itemize}
    \item Context
      \begin{itemize}
      \item secrets?
      \item server?
      \item store?
      \item safely?
      \item \regpg?
      \end{itemize}
    \end{itemize}

    \notes{
      Then I'll give a demo of \regpg's main features, in roughly the same
      order as its reference manual.
    }

    \column{3.5cm}
    \begin{itemize}
    \item Demo
      \begin{itemize}
      \item keys
      \item secrets
      \item X.509 / TLS
      \item Ansible
      \item conversion
      \end{itemize}
    \end{itemize}

  \end{columns}
}

%%%%%%%%%%%%%%%%%%%%%%%%%%%%%%%%%%%%%%%%%%%%%%%%%%%%%%%%%%%%%%%%%%%%%%%%

\section{Context}

\notes{

  In the first part, we'll discuss what \regpg is and what it is not.
}

\slideI[6cm]{keys.jpg}{
  https://www.flickr.com/photos/fuzzy/3196534149
}{secrets}

\notes{

  The secrets we are working with are cryptographic keys
  \begin{itemize}
  \item private keys
  \item bearer tokens
  \item shared secrets
  \end{itemize}

  We have hundreds of them. They need to be shared with the right
  people and kept secret from the wrong people.

  It's a key distribution problem.
}

\slideI[8cm]{encrypt-all-the-things.png}{
}{secrets - encryption}

\notes{

  We can massively reduce the size of the problem by encrypting
  the secrets with a small number of master secrets.

  For example, I used to encrypt secrets using the root password.

  This reduces the key distribution to previously solved problems:

  \begin{itemize}
  \item password distribution
  \item non-secret file distribution (i.e. git)
  \end{itemize}

}

\slideI[10cm]{shamir-rivest-adleman.jpg}{
}{secrets - Shamir / Rivest / Adleman}

\notes{

  But we can do better with public key cryptography.

  Each person keeps their own private key - there's no need to
  distribute any master secrets. We know how to do this because we
  already do it for \ssh keys.

  We distribute the public keys of each person who can decrypt, which
  gives us a kind of auditable record of who has access to secrets.

  We can revoke a person's access if we can destroy all the copies of
  their private key, without having to replace all the secrets.

  You only need the public keys to encrypt a secret, which means an
  automated system can manage its own keys without having access to
  all the other secrets in a repository.

}

\slideI[8cm]{server.jpg}{
  https://www.flickr.com/photos/evilnick/183967344
}{server}

\notes{

  The specific kinds of secrets we are dealing with are used by
  servers to authenticate themselves -

  \begin{itemize}
  \item \ssh host private keys
  \item TLS private keys
  \item API keys
  \item DNS TSIG shared secrets
  \item etc.
  \end{itemize}

  These have to be available unencrypted on the server.

  We're not dealing with user passwords.

  We're not trying to be a password manager.

}

\slideI[9cm]{files.jpg}{
  https://www.flickr.com/photos/lnx/7297731
}{server - files}

\notes{

  It's often the case that each server secret is in a file by itself -
  that's true for \ssh and TLS and DNS keys.

  \regpg works best when each secret is in a file by itself. You can
  use filenames to identify secrets without having to decrypt them.

  Keeping secrets strictly separate from non-secret code and
  configuration helps \git \code{diff} to works better.

  \regpg does not have any hooks into \git for automatically
  decrypting and \code{diff}ing secrets because secrets are blobs of
  random data for which \code{diff} is useless.

}

\slideI[11cm]{library.jpg}{
  https://www.flickr.com/photos/23605686@N05/6921691127
}{store}

\notes{

  \regpg is for encrypting files for storage when they are not in use,
  and decrypting them for deployment to production.

  The other verb that might have fitted in this place is ``share'',
  but \regpg is not directly about sharing.

  \regpg stores secrets in a way that works with \git or other version
  control systems, but \regpg does not get involved with \git. You use
  \git for sharing secrets in the same way you us it for sharing code
  or configuration.

  I have tried a few times to write wrappers that get clever with \git
  and they have usually been dismal failures. \regpg does not get
  clever with \git.

}

\slideI[5cm]{hazmat.jpg}{
  https://www.flickr.com/photos/mamboman/3698344360
}{safely - hazmat containment}

\notes{

  There are a couple of aspects to being safe with \regpg, and both of
  them relate to dissatisfaction with \code{ansible-vault}.

  The first is safe cryptography.

  \regpg keeps well away from any low-level primitives. I did a code
  review of \code{ansible-vault} and it uses a cryptographic library
  that literally has ``HAZMAT'' in its name. And, totally predictably,
  \code{ansible-vault} has really bad crypto.

  Instead, \regpg relies on \gpg for cryptography. \gpg is terrible
  software in many ways, but it is widely available, it has reasonably
  competent crypto, and it is also used by \git and Debian.

}

\slideI[10cm]{checklist.jpg}{
  https://www.flickr.com/photos/109570752@N02/15118828431
}{safely - situational awareness}

\notes{

  The other aspect to being safe is psychological safety.

  \regpg allows you to make it clear in your Ansible playbook which
  files should be encrypted, helps you to find out which files
  actually are or are not encrypted, and tells you when things are
  inconsistent.

  This is unlike \code{ansible-vault} which does not let you say
  whether something should be encrypted, and encourages you to encrypt
  and decrypt in place, and doesn't complain either way, so you can
  easily expose secrets by mistake.

  \regpg tries to be really easy to understand. It isn't very chatty,
  but it also does not hide things from you. I want you to feel
  confident that you know how it works and what it is doing.

}

\slideI[8cm]{re3d.png}{
}{\regpg\ - \url{https://dotat.at/prog/regpg/}}

%%%%%%%%%%%%%%%%%%%%%%%%%%%%%%%%%%%%%%%%%%%%%%%%%%%%%%%%%%%%%%%%%%%%%%%%

\slideI[9cm]{what.jpg}
       {https://www.flickr.com/photos/debord/4932655275}
       {Questions?}

%

\notes{
  \bibliography{talk}
  \bibliographystyle{plain}
}

\end{document}

% eof
